\chapter*{Complementary 1: The Axiom of Choice} \label{sec:}

% \begin{remark}
% The index set $I$ could be countable or uncountable.
% \end{remark}

\begin{axiom}[The Axiom of Choice [AC]]
 \normalfont For any collection $X$ of nonempty sets, there exists a choice function $f: X \to \bigcup X $, such that for every $A \in X$, $f(A) \in A$.
\end{axiom}

\begin{thmintuition}
    AC tells us that for any given collection of non-empty sets, we can pick one element from each set and form a new set.
\end{thmintuition}

\begin{corollary} [Zorn's lemma]
 \normalfont Suppose a partially ordered set $P$ has the property that every chain in $P$ has an upper bound in $P$. Then the set $P$ contains at least one maximal element.
\end{corollary}
\begin{proofidea}
    Assume the contrary, we can build a very long chain that has no upper bound in $P$.
\end{proofidea}


\section*{Banach–Tarski paradox} \label{sec:}

\begin{corollary}[strong form of the Banach–Tarski paradox]
 \normalfont Given any two bounded subsets $A$ and $B$ of a Euclidean space in at least three dimensions, both of which have a nonempty interior, there are partitions of $A$ and $B$ into a finite number of disjoint subsets, $A = A_1 \cup \cdots \cup A_k$, $B = B_1 \cup \cdots \cup B_k$ (for some integer $k$), such that for each (integer) $i$ between 1 and $k$, the sets $A_i$ and $B_i$ are congruent.
\end{corollary}

