\chapter{Preliminary}

\begin{definition}[Isometry]
\normalfont Let $(X,d_X)$ and $(Y, d_Y)$ be metric spaces. A map $f: X \to Y$ is called \textbf{isometry} if for any $a,b \in X$,
\begin{align*}
    d_X(a,b) = d_Y(f(a), f(b)).
\end{align*}
\end{definition}

\begin{definition}[Congruent]
 \normalfont Two subsets $E,F \subseteq \mathbb{R}^{n}$ are called \textbf{congruent} if there exists an isometry $f: \mathbb{R}^{n} \to \mathbb{R}^{n}$ s.t. $f(E) = F$.
\end{definition}

\begin{defintuition}
    $E$ congruent to $F$ if $E$ can be transformed into $F$ by translations, rotations, and reflections.
\end{defintuition}

\begin{definition}[Indexed family]
    \normalfont Let $I$ and $X$ be sets and $f$ a function s.t.
    \begin{align*}
       f : & I \to X
       \\ & i \mapsto x_i = f(i)
    \end{align*}
   (note that we denoted the image of $i$ under $f$ by $x_i$). We then call the image of $I$ under $f$ a \textbf{family of elements in $X$ indexed by $I$}. 
\end{definition}

\begin{remark}
    Similar definition goes to \textbf{indexed collection of nonempty sets}.
\end{remark}

\section{Topology}
\begin{proposition} \label{prop:1.1}
 \normalfont
 \begin{enumerate}
    \item Any open subset of $\mathbb{R}$ is a countable union of disjoint open intervals.
    \item Any open interval is a countable union of closed intervals.
 \end{enumerate}
\end{proposition}

\section{Sequence and series} \label{sec:}

\begin{definition}
 \normalfont If $X$ is an arbitrary set and $f: X \to [0,\infty]$, we define 
 \begin{align*}
    \sum\limits_{x \in X} f(x) = \sup_{} \{ \sum\limits_{x \in F} f(x) : F \subseteq X , |F| < \infty  \}
 \end{align*}
\end{definition}

\begin{remark}
We define the sum to be the supremum of its finite partial sums so that it also works for uncountable set $X$.
\end{remark}

\begin{proposition} \label{1.2.1}
 \normalfont Given $f: X \to [0, \infty]$, let $A = \{ x \in X : f(x) > 0 \}$. If $A$ is uncountable, then $\sum_{x \in X}f(x) = \infty$. If $A$ is countably infinite, then $\sum_{x \in X}f(x) = \sum_{1}^{\infty}f(g(n))$ where $g: \mathbb{N} \to A$ is any bijection and the sum $\sum_{1}^{\infty}f(g(n))$ is an arbitrary infinite series.
\end{proposition}
\begin{proof}
    We can write $A = \bigcup_{n = 1}^{\infty} A_n$, where $A_n = \{  x \in X : f(x) > \frac{1}{n} \}$. If $A$ is uncountable, then there exists $A_k$ is uncountable, thus contains infinite many elements and $\sum_{x \in F} f(x) > |F|/n$ where $F \subset A_k$ is finite. As $|F| \to \infty$, $\sum_{x \in X}f(x) \to \infty$. 
    
    If $A$ is countably infinite, then there exists a bijection $g: \mathbb{N} \to A$. For every finite subset $F$ of $A$, we can always find a set $B_N = g(\{ 1,\cdots ,N \})$ containing $F$ with large enough $N$. Hence
    \begin{align*}
        \sum\limits_{x \in F}^{} f(x) \le \sum\limits_{1}^{N} f(g(n)) \le \sum\limits_{x \in X}^{} f(x)
    \end{align*}
    By taking the supremum over $N$, we have
    \begin{align*}
        \sum\limits_{x \in F}^{} f(x) \le \sum\limits_{1}^{\infty} f(g(n)) \le \sum\limits_{x \in X}^{} f(x)
    \end{align*}
    and then taking the supremum over $F$, we obtain the desired result.
\end{proof}


