\documentclass[11pt]{article}
\usepackage[utf8]{inputenc}	% Para caracteres en español
\usepackage{amsmath,amsthm,amsfonts,amssymb,amscd}
\usepackage{multirow,booktabs}
\usepackage[table]{xcolor}
\usepackage{fullpage}
\usepackage{lastpage}
\usepackage{enumitem}
\usepackage{fancyhdr}
\usepackage{mathrsfs}
\usepackage{wrapfig}
\usepackage{setspace}
\usepackage{calc}
\usepackage{multicol}
\usepackage{cancel}
\usepackage[retainorgcmds]{IEEEtrantools}
\usepackage[margin=3cm]{geometry}
\usepackage{listings}
\usepackage{empheq}
\usepackage{framed}
\usepackage[most]{tcolorbox}
\usepackage{xcolor}

% Pseudo-code
\usepackage{algorithm}
\usepackage[noend]{algpseudocode}
\makeatletter
\def\BState{\State\hskip-\ALG@thistlm}
\makeatother

% Code style
\definecolor{codegreen}{rgb}{0,0.6,0}
\definecolor{codegray}{rgb}{0.5,0.5,0.5}
\definecolor{codepurple}{rgb}{0.58,0,0.82}
\definecolor{backcolour}{rgb}{0.95,0.95,0.92}
\lstdefinestyle{mystyle}{
    backgroundcolor=\color{backcolour},   
    commentstyle=\color{codegreen},
    keywordstyle=\color{magenta},
    numberstyle=\tiny\color{codegray},
    stringstyle=\color{codepurple},
    basicstyle=\ttfamily\footnotesize,
    breakatwhitespace=false,         
    breaklines=true,                 
    captionpos=b,                    
    keepspaces=true,                 
    numbers=left,                    
    numbersep=5pt,                  
    showspaces=false,                
    showstringspaces=false,
    showtabs=false,                  
    tabsize=2
}
\lstset{style=mystyle}

% Unclassified
\newlength{\tabcont}
\setlength{\parindent}{0.0in}
\setlength{\parskip}{0.05in}
\colorlet{shadecolor}{orange!15}
\parindent 0in
\parskip 12pt
\geometry{margin=1in, headsep=0.25in}
\theoremstyle{definition}
\newtheorem{defn}{Definition}
\newtheorem{reg}{Rule}
\newtheorem{exer}{Exercise}
\newtheorem{note}{Note}
\newtheorem{theorem}{Theorem}
\newtheorem{lemma}{lemma}
\newtheorem{remark}{remark}
\newtheorem{corollary}{corollary}
\begin{document}

\setcounter{section}{-1}

\title{Decision Tree}

\thispagestyle{empty}

\begin{center}
{\LARGE \bf Decision Tree}\\
{\large Xuzhe Xia}\\
\today
\end{center}

% The document BEGIN here
% ------------------------------------------------------------------------------------------------------
  
\section{Definitions} \label{sec:}

Decision tree is:
\begin{itemize}
    \item A nested sequence of "if-else" decision based on the features.
    \item A class as a return value at the end of each sequence.
\end{itemize}

Decision stump is:
\begin{itemize}
    \item A simple decision tree with 1 splitting rule based on threshold 1 feature.
    \item To learn a decision stump we need to know: which feature use to split; what value to used; classes assigned to leaves.
\end{itemize}

Notations:
\begin{itemize}
    \item Feature matrix $X$: $n \times d$ dimension. ($n$: number of samples, $d$: number of features).
    \item Label vector $y$: each entry is a label of corresponds sample.
\end{itemize}


\section{Accuracy-based decision tree} \label{sec:}

\subsection{Decision Stump} \label{sec:}

Choice of decision stump:
\begin{itemize}
    \item Choose the best stump based on accuracy.
    \item 
\begin{lstlisting}[language=Python, caption=Decision stmp learning pseudo-code]
for feature "j":
    for each threshold "t":
        set "y-yes" to most common label of objects satisfying rule.
        set "y-no" to most common label of objects not satisfying rule.
        set "y-hat" to most common label of objects "i" satisfying rule.
        compute error = number of objects where y != y-hat.
\end{lstlisting}
    \item complexity: 
\end{itemize}

% \begin{algorithm}
%     \caption{My algorithm}\label{euclid}
%     \begin{algorithmic}[1]
%         \Procedure{MyProcedure}{}
%         \EndProcedure
%     \end{algorithmic}
% \end{algorithm}


% ------------------------------------------------------------------------------------------------------
% The document ENDS here

\end{document}