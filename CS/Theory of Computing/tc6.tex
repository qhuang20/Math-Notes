\setcounter{chapter}{1}
\chapter{Communication complexity}


\begin{definition}[Deterministic Communication Complexity]
For $f: X \times Y \to Z$, the \textit{Deterministic Communication Complexity} is
\begin{align*}
    D(f) = \min_{\text{protocal computing} f} \{ \max_{x \in X, y \in Y} \{ \text{number of bits on input (x,y)} \} \}
    \\ = \min_{\text{protocal tree } T \text{ for } f} \text{depth}(T)
 \end{align*}
\end{definition}

\begin{observe}
    For any arbitrary tree, number of leaves $\le 2^{\text{depth}}$.
\end{observe}

\begin{definition}[Rectangle]
A \textit{rectangle} in $X \times Y$ is a set of the form $S = A \times B$ where $A \subseteq X$ and $B \subseteq Y$.
\end{definition}

\begin{observe}
    $X$ is a rectangle if and only if
    \begin{align*}
        \left( (x,y) \in S \land (x',y')\in S \right) \implies \left( (x,y') \in S \land (x',y)\in S \right)
    \end{align*}
\end{observe}

\begin{remark}
Let $v$ be a node in a protocol tree. Let $R_v$ be inputs that arrives at $v$, then $R_v$ is a rectangle.
\end{remark}

\begin{definition}[f-monochromatic]
A rectangle $R$, is \textit{f-monochromatic} if $\exists z \in Z$ s.t.t $\forall (x,y) \in R$, $f(x,y)=z$.
\end{definition}

\begin{remark}
If $v$ is a leaf, then $R_v$ is an f-monochromatic rectangle.
\end{remark}

\begin{remark} ~
Every input $(x,y)$ must arrive at exactly one leaf $v$. So $(x,y)\in R_v$. 
\end{remark}

\begin{prop}
The set $\{ R_v : v \text{ is a leaf of } T \}$ is a partition of $X \times Y$ into f-monochromatic rectangles.
\end{prop}

\begin{definition}[C-partition]
$C^{\text{partition}}(f) = \min\{ |\mathcal{R}| : \mathcal{R} \text{ is a f-monochromatic partition }\}$.
\end{definition}

$C^{\text{partition}}(f)$ is the minimum number of rectangles needed in any f-monochromatic partition.

\begin{corollary}
For any protocol tree $T$, $C^{\text{partition}}(f) \le $ number of leaves in $T$.
\end{corollary}

\begin{corollary}
$C^{\text{partition}}(f) \le \min_{\text{protocal tree}}\{ 2^{\text{depth}}(T)\}$
\end{corollary}

\begin{theorem}
    \begin{align*}
        \left\lceil \log_2 C^{\text{partition}}(f) \right\rceil  \le \min_{\text{protocal tree}}\{ \text{depth}(T)\} = D(T)
    \end{align*}
\end{theorem}

\begin{definition}[Fooling set]
TODO
\end{definition}

\begin{prop}
\begin{align*}
    C^{\text{partition}}(f) \ge |S| + 1
\end{align*}
unless $f$ is constant
\end{prop}

\begin{corollary}
    \begin{align*}
        D(f) \ge \left\lceil \log_2 (|S| + 1) \right\rceil 
    \end{align*}
\end{corollary}

\begin{eg}
    Let $EQ_n : \{ 0,1 \}^{n} \times \{ 0,1 \}^{n} \to \{ 0 , 1 \} $
    \begin{align*}
        EQ_n (x,y) = \begin{cases}
            1 & \text{ if } x = y \\
            0 & \text{o.w.} \\
        \end{cases}
    \end{align*}
    The upper bound: $D(EQ_n) \le n + 1$
    \\The fooling set is $S = \{ (x,x) : x \in \{ 0,1 \}^{n} \}$, since:
    \begin{enumerate}
        \item $EQ_n(x,y) = 1, \forall x \in \{ 0,1 \}^{n}$.
        \item Consider $(x,x)$ and $(x',x') \in S$ where $x \neq x'$, we have $EQ_n(x,x') = EQ_n(x',x) = 0$.
    \end{enumerate}
    Conclude: $D(EQ_n) \ge \left\lceil \log_2(2^{n} + 2) \right\rceil = n + 1$
\end{eg}

\begin{definition}[Nondeterministic communication complexity]
\textit{Nondeterministic communication complexity} of $f$ is $N(f)$
    \begin{align*}
        = \min_{\text{nondet. protocal}} \{ \max_{x, y, \text{ s.t. } f(x,y) = 1} \{ \text{number of bits on a certificate for input }x,y \} \}
    \end{align*}
\end{definition}

\begin{eg}
    For $f = EQ_n$:
    \\ \textbf{Protocol 1:} 
    \begin{enumerate}
        \item Prover lets $z = (x,y)$
        \item Alice accepts if $x = z[1] = z[2]$
        \item Bob accepts if $x = z[2] = z[1]$
    \end{enumerate}
    In this case, $z$ is of length $2n$.
    \\ \textbf{Protocol 2:} 
    \begin{enumerate}
        \item Prover lets $z = x$
        \item Alice accepts if $z = y$
        \item Bob accepts if $z = y$
    \end{enumerate}
    (Explanation: If $x = y$, Prover sets $z=x$, both accepts; otherwise, there is no string $z$ that will make both accepts.)
    \\In this case, $z$ is of length $n$.
\end{eg}
\begin{remark}
    \begin{align*}
        N(EQ_n) \le n
    \end{align*}
\end{remark}

\begin{eg}
    For $f = \neg EQ_n$:
    \\ \textbf{Protocol 1:} silly protocol, in this case, $z$ is of length $2n$.
    \\ \textbf{Protocol 2:} 
    \begin{enumerate}
        \item Prover lets $z = (i,b)$
        \item Alice accepts if $x_i \neq b$
        \item Bob accepts if $y_i = b$
    \end{enumerate}
    (Explanation: If $x \neq y$, Prover finds $i$ s.t. $x_i \neq y_i$, lets $b = y_i$, both will accepts; otherwise if $x = y$, one of them must reject regardless of $z$.)
    \\In this case, $z$ is of length $\left\lceil \log(n) \right\rceil + 1$.
\end{eg}
\begin{remark}
    \begin{align*}
        coN(EQ_n) = N(\neg EQ_n) \le \left\lceil \log(n) \right\rceil + 1
    \end{align*}
\end{remark}

\subsection{Cover} \label{sec:}

\begin{definition}[Cover]
    The set $R = \{ R_1,\cdots ,R_n \}$ is a 1-cover of the 1-entries if:
    \begin{enumerate}
        \item Each $R_i$ is a rectangle where all values are all equal to 1.
        \item Each $(x,y)\in X\times Y$ with $f(x,y)=1$ is contained in at least one $R_i$ (Needn't be disjoint)
    \end{enumerate}
\end{definition}

\begin{definition}
    \begin{align*}
        C^{\text{1-cover}}(f) = \min\{ |R| : R \text{ is a cover of the 1-entries} \}
    \end{align*}
    \begin{align*}
        C^{\text{0-cover}}(f) = \min\{ |R| : R \text{ is a cover of the 0-entries} \} = C^{\text{1-cover}}(\neg f)
    \end{align*}
\end{definition}

\begin{observe}
    \begin{align*}
        C^{\text{partition}}(f) \ge C^{\text{1-cover}}(f) + C^{\text{0-cover}}(f)
    \end{align*}
\end{observe}

\begin{remark}
    \begin{align*}
        C^{\text{0-cover}}(DIST_n) \le n
    \end{align*}
    \begin{align*}
        C^{\text{0-cover}}(EQ) \le 2n
    \end{align*}
\end{remark}

\begin{previouslyseen}
    \begin{align*}
        D(f) \ge \left\lceil \log_2(C^{\text{partition}}(f)) \right\rceil 
    \end{align*}
\end{previouslyseen}

\begin{theorem}
    \begin{align*}
        N(f) = \left\lceil \log_2(C^{\text{1-cover}}(f)) \right\rceil 
    \end{align*}
    \begin{align*}
        coN(f) = \left\lceil \log_2(C^{\text{0-cover}}(f)) \right\rceil 
    \end{align*}
\end{theorem}

\begin{corollary}
\begin{align*}
    D(f) \ge N(f)
    \\ D(f) \ge coN(f)
\end{align*}
\end{corollary}

\begin{observe}
    Deterministic communication complexity is closed under complement, but Nondeterministic communication complexity does not.
\end{observe}
